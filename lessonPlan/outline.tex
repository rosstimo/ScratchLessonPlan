
\documentclass[main.tex]{subfiles}
%%------ Preamble specific to this subfile only. This will be used when this document is compiled on its own. When the main document compiles this preabble is ignored. 
%%---------------------------------------------------------------------------%%
\begin{document}

\section*{5-Day, 4 Hours per Day Lesson Plan: Learn Programming and Make a Video with Scratch}

\subsection*{Day 1: Introduction to Scratch and Basic Concepts}

\paragraph{Hour 1: Introduction to Scratch}
\begin{itemize}
  \item Overview of Scratch: What it is, its purpose, and examples of projects.
  \item Create Scratch accounts.
  \item Brief tour of the Scratch interface.
\end{itemize}

\paragraph{Hour 2: Basic Programming Concepts}
\begin{itemize}
  \item Introduce sprites, stages, and the concept of scripts.
  \item Basic Scratch blocks: Motion, Looks, Sound.
  \item Simple exercise: Make a sprite move and say something.
\end{itemize}

\paragraph{Hour 3: Creating Simple Animations}
\begin{itemize}
  \item Use motion and looks blocks to create simple animations.
  \item Exercise: Create an animation where a sprite moves across the screen and changes costumes.
\end{itemize}

\paragraph{Hour 4: Introduction to Control Blocks}
\begin{itemize}
  \item Explain control blocks (e.g., loops, if-then statements).
  \item Exercise: Create a sprite that repeats an action using loops.
\end{itemize}

\subsection*{Day 2: Advanced Scratch Blocks and Interactive Projects}

\paragraph{Hour 1: More on Control Blocks and Variables}
\begin{itemize}
  \item Deep dive into control blocks and introduction to variables.
  \item Exercise: Create a project that uses loops and variables.
\end{itemize}

\paragraph{Hour 2: Introduction to Events and Sensing Blocks}
\begin{itemize}
  \item Explain events and how to use sensing blocks to interact with the user.
  \item Exercise: Create an interactive project where a sprite reacts to keyboard inputs.
\end{itemize}

\paragraph{Hour 3: Introduction to Operators and More Complex Interactions}
\begin{itemize}
  \item Discuss operators and how to use them in Scratch.
  \item Exercise: Create a project that includes conditional statements using operators.
\end{itemize}

\paragraph{Hour 4: Planning Your Video Project}
\begin{itemize}
  \item Brainstorming session: Ideas for a short video project.
  \item Start planning: Storyboarding and listing necessary sprites and actions.
\end{itemize}

\subsection*{Day 3: Developing the Video Project}

\paragraph{Hour 1: Creating and Importing Sprites}
\begin{itemize}
  \item How to create and customize sprites.
  \item Importing and using existing sprites from the Scratch library.
  \item Exercise: Create custom sprites for your video project.
\end{itemize}

\paragraph{Hour 2: Designing Backgrounds and Scenes}
\begin{itemize}
  \item How to create and switch between backgrounds.
  \item Exercise: Design and implement backgrounds for different scenes of your video.
\end{itemize}

\paragraph{Hour 3: Scripting Actions and Dialogues}
\begin{itemize}
  \item Writing scripts for sprite actions and dialogues.
  \item Exercise: Script the first part of your video project.
\end{itemize}

\paragraph{Hour 4: Adding Sound and Music}
\begin{itemize}
  \item How to add and customize sound effects and music.
  \item Exercise: Add appropriate sound effects and background music to your video.
\end{itemize}

\subsection*{Day 4: Enhancing the Video Project}

\paragraph{Hour 1: Refining Animations and Transitions}
\begin{itemize}
  \item Smooth out animations and transitions between scenes.
  \item Exercise: Enhance the animations and transitions in your video project.
\end{itemize}

\paragraph{Hour 2: Implementing Interactivity}
\begin{itemize}
  \item Adding interactive elements to your video (e.g., clickable buttons, user input).
  \item Exercise: Incorporate interactive elements into your project.
\end{itemize}

\paragraph{Hour 3: Testing and Debugging}
\begin{itemize}
  \item Test the project for any bugs or issues.
  \item Debugging techniques and tips.
  \item Exercise: Test and debug your video project.
\end{itemize}

\paragraph{Hour 4: Finalizing the Project}
\begin{itemize}
  \item Final adjustments and polishing.
  \item Review and finalize the video project for presentation.
\end{itemize}

\subsection*{Day 5: Presentation and Sharing}

\paragraph{Hour 1: Project Presentation}
\begin{itemize}
  \item Each student presents their video project to the class.
  \item Provide constructive feedback on each project.
\end{itemize}

\paragraph{Hour 2: Peer Review and Improvement}
\begin{itemize}
  \item Students review each other’s projects and suggest improvements.
  \item Time for making final adjustments based on feedback.
\end{itemize}

\paragraph{Hour 3: Exporting and Sharing Projects}
\begin{itemize}
  \item How to save, export, and share Scratch projects.
  \item Exercise: Export your project and upload it to the Scratch community.
\end{itemize}

\paragraph{Hour 4: Celebration and Reflection}
\begin{itemize}
  \item Celebrate the completion of the projects with a showcase.
  \item Reflect on the learning experience and discuss what students enjoyed and what could be improved.
\end{itemize}


\section*{Day 1: Introduction and Game Setup}
\subsection*{Hour 1: Welcome and Introduction}
\begin{itemize}
    \item Introduction to Scratch and camp goals.
    \item Show examples of simple scrolling/platform games.
\end{itemize}

\subsection*{Hour 2: Game Concept and Design}
\begin{itemize}
    \item Discuss game elements: intro animation, player character, NPCs, power-ups, score, lives, game over.
    \item Sketch the game design on paper.
\end{itemize}

\subsection*{Hour 3: Setting Up the Game Stage}
\begin{itemize}
    \item Create the game stage (background, scrolling effect).
    \item Introduce motion blocks for background scrolling.
    \item Hands-on activity: Students create their game stage.
\end{itemize}

\subsection*{Hour 4: Creating the Player Character}
\begin{itemize}
    \item Add and animate the player character sprite.
    \item Introduce basic motion (move left, right, jump).
    \item Hands-on activity: Students program their player character's movement.
\end{itemize}

\section*{Day 2: Basic Game Mechanics}
\subsection*{Hour 1: Intro Animation and Menu}
\begin{itemize}
    \item Create an intro animation and main menu.
    \item Introduce event blocks for starting the game.
    \item Hands-on activity: Students create and program their game menus.
\end{itemize}

\subsection*{Hour 2: Player Character Animation}
\begin{itemize}
    \item Refine player character animations (walking, jumping).
    \item Introduce loops and conditional statements.
    \item Hands-on activity: Students enhance their character animations.
\end{itemize}

\subsection*{Hour 3: Power-Up Interactions}
\begin{itemize}
    \item Add power-up sprites and interactions.
    \item Introduce sensing blocks for detecting collisions with power-ups.
    \item Hands-on activity: Students program power-up effects.
\end{itemize}

\subsection*{Hour 4: Non-Player Characters (NPCs)}
\begin{itemize}
    \item Add NPC sprites and animations.
    \item Program NPC movement and behaviors.
    \item Hands-on activity: Students create and program NPCs.
\end{itemize}

\section*{Day 3: Advanced Interactions and Scoring}
\subsection*{Hour 1: Collisions and Interactions}
\begin{itemize}
    \item Detect collisions between player and obstacles/enemies.
    \item Introduce control blocks for handling collisions.
    \item Hands-on activity: Students program collision detection.
\end{itemize}

\subsection*{Hour 2: Keeping Score}
\begin{itemize}
    \item Introduce variables for score tracking.
    \item Program score increment on collecting items or defeating NPCs.
    \item Hands-on activity: Students implement score tracking.
\end{itemize}

\subsection*{Hour 3: Tracking Player Lives}
\begin{itemize}
    \item Introduce variables for tracking player lives.
    \item Program lives decrement on collisions with enemies/obstacles.
    \item Hands-on activity: Students add life tracking to their games.
\end{itemize}

\subsection*{Hour 4: Game Over Trigger}
\begin{itemize}
    \item Detect when player lives reach zero.
    \item Program game over condition and transition.
    \item Hands-on activity: Students implement game over conditions.
\end{itemize}

\section*{Day 4: Enhancing the Game}
\subsection*{Hour 1: Game Over Animation}
\begin{itemize}
    \item Create a game over animation sequence.
    \item Program the transition to game over screen.
    \item Hands-on activity: Students design and implement game over animations.
\end{itemize}

\subsection*{Hour 2: Adding Levels}
\begin{itemize}
    \item Plan and design additional levels for the game.
    \item Introduce broadcast messages for level transitions.
    \item Hands-on activity: Students create and program level transitions.
\end{itemize}

\subsection*{Hour 3: Debugging and Testing}
\begin{itemize}
    \item Test the game for bugs and issues.
    \item Introduce debugging techniques.
    \item Hands-on activity: Students test and debug their games.
\end{itemize}

\subsection*{Hour 4: Mid-Project Check-In}
\begin{itemize}
    \item Teams present their progress and receive feedback.
    \item Discuss improvements and next steps.
\end{itemize}

\section*{Day 5: Finalizing and Presenting}
\subsection*{Hour 1: Finalizing Projects}
\begin{itemize}
    \item Complete all remaining game features.
    \item Conduct final testing and debugging.
\end{itemize}

\subsection*{Hour 2: Preparing Presentations}
\begin{itemize}
    \item Guide teams in preparing their presentations.
    \item Practice presenting their projects.
\end{itemize}

\subsection*{Hour 3: Project Presentations}
\begin{itemize}
    \item Each team presents their project to the class.
    \item Use a rubric to evaluate their work.
\end{itemize}

\subsection*{Hour 4: Wrap-Up and Reflection}
\begin{itemize}
    \item Discuss what students learned during the camp.
    \item Provide certificates and awards.
    \item Final Q&A and feedback session.
\end{itemize}

\end{document}
    
